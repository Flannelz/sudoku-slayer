\documentclass[10pt, conference]{IEEEtran}

\usepackage{cite}
\usepackage{amsmath,amssymb,amsfonts}
\usepackage{algorithmic}
\usepackage{graphicx}
\usepackage{textcomp}
\usepackage{xcolor}
\def\BibTeX{{\rm B\kern-.05em{\sc i\kern-.025em b}\kern-.08em
    T\kern-.1667em\lower.7ex\hbox{E}\kern-.125emX}}


\begin{document}

\title{Sudoku Slayer}

\author{\IEEEauthorblockN{Alice Hanigan}
\and
\IEEEauthorblockN{John Summers}
}

\maketitle

\begin{abstract}

The use of Constraint Satisfaction Problems(CSP) for modeling and solving Sudoku Puzzles is a natural implimentation of CSP techniques.
Genetic Algorithms(GA) offer a much broader approach to the search space, but are not traditionally employed to solve Sudoku puzzles due to the puzzles structure naturally lending itself to other algorithms.
The size of a Sudoku's search space is considerable and because of this, we decided to compare the use of CSP and GA's for solving Sudoku puzzles.

\end{abstract}

\section{introduction}

\section{Literature Review}

\section{Methodology}

\section{Results}

\section{Discussion}

\section{Conclusion}

\begin{thebibliography}{00}
\bibitem{b1}J. Doe, ``Example Paper,'' unpublished
\end{thebibliography}

\end{document}