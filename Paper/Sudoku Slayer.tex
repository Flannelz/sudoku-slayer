\documentclass[10pt, conference]{IEEEtran}

\usepackage{cite}
\usepackage{amsmath,amssymb,amsfonts}
\usepackage{algorithmic}
\usepackage{graphicx}
\usepackage{textcomp}
\usepackage{xcolor}
\def\BibTeX{{\rm B\kern-.05em{\sc i\kern-.025em b}\kern-.08em
    T\kern-.1667em\lower.7ex\hbox{E}\kern-.125emX}}


\begin{document}

\title{Sudoku Slayer}

\author{\IEEEauthorblockN{Alice Hanigan}
\and
\IEEEauthorblockN{John Summers}
}

\maketitle

\begin{abstract}

The use of Constraint Satisfaction Problems(CSP) for modeling and solving Sudoku Puzzles is a natural implimentation of CSP techniques.
Genetic Algorithms(GA) offer a much broader approach to the search space, but are not traditionally employed to solve Sudoku puzzles due to the puzzles structure naturally lending itself to other algorithms.
The size of a Sudoku's search space is considerable and because of this, we decided to compare the use of CSP and GA's for solving Sudoku puzzles.

\end{abstract}

\section{introduction}

\section{Literature Review}

\subsection{Sudoku as a Constraint Problem}

\subsection{Complexity and Completeness of Finding another Solution and its Application to Puzzles}

\subsection{Review of Selection Methods in Genetic Algoritms}

\subsection{Multi-Parent Recombination}

\section{Methodology}

\section{Results}

\section{Discussion}

\section{Conclusion}

\begin{thebibliography}{00}

\bibitem{b1}Simonis, Helmut, 2006, "Sudoku as a Constraint Problem", http://citeseerx.ist.psu.edu/viewdoc/download?doi=10.1.1.88.2964\&rep=rep1\&type=pdf
\bibitem{b2}Tayo, Takayuki, “Complexity and Completeness of Finding Another Solution and Its Application to Puzzles”, Takahiro Seta, http://www-imai.is.s.u-tokyo.ac.jp/~yato/data2/SIGAL87-2.pdf
\bibitem{b3}Nisha Saini, "review of Selection Methods in Genetic Algorithms", https://www.ijecs.in/index.php/ijecs/article/download/2562/2368/
\bibitem{b4}A. E. Eiben, “Multi-Parent Recombination”, https://www.cs.vu.nl/~gusz/papers/Handbook-Multiparent-Eiben.pdf

\end{thebibliography}

\end{document}